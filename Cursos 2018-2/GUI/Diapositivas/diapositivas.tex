\documentclass[11pt]{beamer}
\usetheme{Copenhagen}
\usepackage[spanish]{babel}
\usepackage[utf8]{inputenc}
\usepackage{graphicx}
\usepackage{tikz}
\usetikzlibrary{arrows,decorations.pathmorphing,backgrounds,fit,petri,positioning,matrix}
\usepackage{listings}


\author{J. Eduardo Sánchez Posadas}
\title{Interfaces Gráficas de Usuario con Java}
%\setbeamercovered{transparent} 
%\setbeamertemplate{navigation symbols}{} 
%\logo{} 
\institute{FES Aragón} 
 
\subject{GUI con Java} 
\begin{document}

\begin{frame}
\titlepage
\end{frame}

\begin{frame}
\tableofcontents
\end{frame}

\section{Objetivos}
\begin{frame}{Objetivos}
\begin{itemize}
\item Diseñar interfaces gráficas
\item Conocer el entorno de interfaces gráficas de Java
\end{itemize}
\end{frame}

%\section{Justificación}
%\begin{frame}{Justificación}
%\begin{itemize}
%\item 
%\end{itemize}
%\end{frame}

\section{Temario}
\begin{frame}{Temario}
\begin{enumerate}
\item Contenedores
\item Componentes
\item Cuadros de Diálogo
\item Manejadores de composicion
\item Formularios
\item Manejo de eventos
\item Menús
\item Gráficos
\item Applets*
\end{enumerate}
\end{frame}



\section{Evaluación}
\begin{frame}{Evaluación}
\begin{itemize}
\item Mínimo 80\% de asistencia
\item Actividades: 30\%
\item Proyecto: 70\%
\end{itemize}
\end{frame}

\section{Contenedores}
\begin{frame}{Contenedores}
\begin{itemize}
\item Elemento gráfico
\item Permite agrupar otros elementos
\item Debe haber al menos uno por aplicación para iniciarise
\end{itemize}
\end{frame}

\subsection{JFrame}
\begin{frame}{\textbf{JFrame}}
Un JFrame es un contenedor que se comporta como una ventana, la cual puede tener propiedades físicas. Estas propiedades pueden estar dadas por el tamaño, color y posición, entre otras.
\end{frame}

\begin{frame}[fragile]{MiJFrame}
\lstset{language=Java,
                basicstyle=\ttfamily,
                keywordstyle=\color{blue}\ttfamily,
                stringstyle=\color{red}\ttfamily,
                commentstyle=\color{green}\ttfamily,
                morecomment=[l][\color{magenta}]{\#},
                numbers=left
}
\begin{tiny}

\begin{lstlisting}
package gui.contenedores;

import javax.swing.JFrame;
import javax.swing.WindowConstants;

public class MiJFrame extends JFrame {

    public static void main(String[] args) {
        MiJFrame frame = new MiJFrame();
        frame.setVisible(true);
    }

    public MiJFrame() {
        initGUI();
    }

    private void initGUI() {
        setDefaultCloseOperation(WindowConstants.DISPOSE_ON_CLOSE);
        setTitle("Mi JFrame");
        setSize(400, 300);
    }
}

\end{lstlisting}
\end{tiny}
\end{frame}

\begin{frame}

\begin{center}
\begin{figure}
\includegraphics[scale=0.2]{pics/pic1.png}
\caption{MiJFrame} 
\end{figure}
\end{center}
\end{frame}
\subsection{JInternalFrame}
\begin{frame}{JInternalFrame}
Un \textit{JInternalFrame} es un contenedor que se comporta como una ventana interna, es decir, una ventana que se puede abrir solo dentro de un \textit{JFrame}.
\end{frame}

\begin{frame}
A través de un \textit{JInternalFrame} es posible implementar aplicaciones \textbf{MDI} (Multiple Interface Document), debido a que es posible abrir varios \textit{JInternalFrame} dentro de un \textit{JFrame}, en donde cada uno de ellos provee funcionalidades independientes a la aplicación.
\end{frame}
 
\begin{frame}[fragile]

Para que un \textit{JFrame} pueda contener un \textit{JInternalFrame} es necesario tener dentro del \textit{JFrame} otro contenedor especial denominado \textit{JDesktopPane}. Este contenedor se adiciona al \textit{JFrame} con la siguiente sintaxis:

\lstset{language=Java,
                basicstyle=\ttfamily,
                keywordstyle=\color{blue}\ttfamily,
                stringstyle=\color{red}\ttfamily,
                commentstyle=\color{green}\ttfamily,
                morecomment=[l][\color{magenta}]{\#}
}

\begin{lstlisting}
JDesktopPane desktopPane = new JDesktopPane();
getContentPane().add(desktopPane);

\end{lstlisting}

\end{frame}

\begin{frame}[fragile]
Desde el \textit{JFrame} debe crearse un instancia al \textit{JInternalFrame}. Para adicionar la instancia al \textit{JFrame} se debe hacer a través del \textit{JDesktopPane} con la siguiente sintaxis:

\lstset{language=Java,
                basicstyle=\ttfamily,
                keywordstyle=\color{blue}\ttfamily,
                stringstyle=\color{red}\ttfamily,
                commentstyle=\color{green}\ttfamily,
                morecomment=[l][\color{magenta}]{\#}
}

\begin{lstlisting}
Flnterno frame = new FInterno();
desktopPane.add(frame);
\end{lstlisting}
\end{frame}

\begin{frame}

\begin{center}
\begin{figure}
\includegraphics[scale=0.2]{pics/pic2.png} 
\end{figure}
\end{center}
\end{frame}
\subsection{JPanel}
\begin{frame}{JPanel}
Un \textit{JPanel} es un contenedor que tiene muchas aplicaciones. Dentro de las aplicaciones más comunes están, el permitir agregar componentes para que puedan ser organizados gráficamente de una forma determinada.
\end{frame}

\begin{frame}
Otra aplicación común es utilizar el JPanel como pizarra para gráficos. Un panel también puede tener un título de acuerdo al uso que se le esté dando.
\end{frame}

\begin{frame}{Pila}
\begin{center}
\begin{tikzpicture}[draw, minimum width=1cm, minimum height=0.5cm]
    \node[draw] (in) at (-1,2) {In};
    \node[draw] (out) at (1,2) {Out};
    \matrix (queue) [matrix of nodes, nodes={draw, nodes={draw}}, nodes in empty cells]
    {
       a \\ e \\ \\ \\
    };

    \draw[-latex] (0.25,1) .. controls (0.25,1.5) and (1,1.5) .. (out.south);
    \draw[-latex] (in.south) .. controls (-1, 1.5) and (-0.25,1.5) .. (-0.25,1);
\end{tikzpicture}
\end{center}

\end{frame}

\end{document}
