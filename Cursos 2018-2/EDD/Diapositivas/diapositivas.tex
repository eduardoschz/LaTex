\documentclass[11pt]{beamer}
\usetheme{CambridgeUS}
\usepackage[utf8]{inputenc}
\usepackage[spanish]{babel}
\usepackage{amsmath}
\usepackage{amsfonts}
\usepackage{amssymb}
\usepackage{graphicx}
\usepackage{tikz}



\author{J. Eduardo Sánchez Posadas}
\title{Estructuras de Datos con Java}
%\setbeamercovered{transparent} 
%\setbeamertemplate{navigation symbols}{} 
\logo{\includegraphics[scale=0.5]{pics/g10368.png} } 
%\institute{} 
%\date{} 
%\subject{} 
\begin{document}

\begin{frame}
\titlepage
\end{frame}

\begin{frame}
\tableofcontents
\end{frame}

\section{Objetivos}
\begin{frame}{Evaluación}
\begin{itemize}
\item Minimo 80\% asistencia
\item 40\% tareas y prácticas
\item 60\% proyecto

\end{itemize}
\end{frame}


\begin{frame}{Tipos de estructuras}
\begin{tiny}
\begin{tikzpicture}[level distance=0cm,
   level 1/.style={sibling distance=6cm, level distance=1cm},
   level 2/.style={sibling distance=2cm, level distance=1.25cm},
   level 3/.style={sibling distance=1.1cm, level distance=1cm},
   level 4/.style={sibling distance=1.15cm, level distance=1.5cm},
   level 5/.style={sibling distance=1.125cm, level distance=1.5cm}]
\node {Estructuras de Datos}
   child {
	node {Estrcuturas Simples}
	   child {node {Estructura}
			child{node{\texttt{byte}}}
			child{node{\texttt{short}}}
			child{node{\texttt{int}}}
			child{node{\texttt{long}}}
			child{node{\texttt{float}}}
			child{node{\texttt{\ldots}}}   
	   }
	   child {node {Arreglo}}
}
child {node {Estructuras Compuestas}
	child {node {Lineales}
			child{node{Lista}
			child{node{Simple}}
			child{node{Ligada}}
			child{node{Circular}}
			child{node{Circular doble}}
}
		child{node{Pila}}
		child{node{Cola}}
}
   child {node {No-Lineal}
		child{node{Grafo}
			child{node{Arbol}
				child{node{B}}
				child{node{B+}}}
}
	}
};
\end{tikzpicture}
\end{tiny}

\end{frame}

\end{document}